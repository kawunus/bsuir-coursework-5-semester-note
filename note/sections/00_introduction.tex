\sectioncentered*{Введение}
\addcontentsline{toc}{section}{Введение}

Актуальность создания автоматизированной информационной системы (АИС) студии \textit{3D}-печати обусловлена стремительным развитием аддитивных технологий и ростом спроса на услуги индивидуального производства. В условиях увеличения потока заказов и расширения номенклатуры используемых материалов, ручное управление ключевыми бизнес-процессами, такими как обработка заказов, учет расходных материалов (филамента) и анализ финансовых показателей, становится неэффективным.

Проблема неавтоматизированных систем заключается в повышенном риске человеческой ошибки при расчете стоимости печати, потере данных о статусах заказов, а также в отсутствии оперативной аналитики для принятия управленческих решений.

Хаотичность в управлении производственными процессами и клиентскими заявками приводит к:
\begin{itemize}
    \item некорректному расчету остатков филамента (\textit{residue}), что может вызвать остановку печати из-за нехватки материала;
    \item ошибкам в очередности выполнения заказов и простоям оборудования (\textit{printers});
    \item отсутствию прозрачной истории действий сотрудников и изменений статусов заказов (\textit{order history});
    \item невозможности оценки эффективности рекламных каналов и динамики продаж из-за отсутствия сводных отчетов.
\end{itemize}

Ввиду этих сложностей, внедрение структурированной и надежной информационной системы в виде \textit{web}-приложения становится необходимостью. При автоматизированном подходе значительно сокращаются издержки на брак, повышается точность учета материалов (вплоть до граммов), а руководство получает инструменты для стратегического планирования.

Целью курсовой работы является разработка базы данных и функциональной модели АИС студии \textit{3D}-печати. Система предоставит пользователям простоту организации процесса взаимодействия с услугами студии. Клиенты получат возможность оформления заказов \textit{online}, просмотра актуальных типов пластика и контроля статуса выполнения своих заявок в реальном времени. Сотрудники обретут гибкий механизм для централизованного управления заказами и принтерами. Аналитики (\textit{Analysts}) получат доступ к визуализированной статистике, графикам и отчетам для оценки эффективности работы студии.

Ясность структурированной организации, основанной на ролевой модели (\textit{Client}, \textit{Employee}, \textit{Analyst}, \textit{Admin}), позволит каждому участнику эффективно выполнять свои задачи: от оформления заявки и производства изделия до глубокого анализа рыночного спроса и администрирования системы.

В рамках курсового проекта планируется получение работоспособного программного продукта, способного поддерживать организацию полного цикла предоставления и потребления услуг \textit{3D}-печати, начиная от регистрации пользователя и заканчивая управлением историей заказов и генерацией аналитических отчетов.