\sectioncentered*{Заключение}
\addcontentsline{toc}{section}{Заключение}

В результате выполнения курсового проекта разработана автоматизированная информационная система (АИС) студии \textit{3D}-печати для управления заказами, производственным оборудованием, складским учетом материалов и бизнес-аналитикой.

В процессе работы был проведен всесторонний анализ предметной области, что позволило сформировать четкие требования к системе и определить её практическое применение в условиях современного рынка аддитивных технологий. Для уточнения структуры и функционала системы были изучены примеры существующих решений, что помогло выделить ключевые компоненты для ее эффективного функционирования, включая необходимость расширенной ролевой модели (клиент, сотрудник, аналитик, администратор).

Курсовой проект подробно описывает этапы разработки, начиная с проектирования логической структуры базы данных, определения функциональности для каждой роли и заканчивая программной реализацией.

Клиентская часть системы (\textit{Frontend}) была выполнена с использованием библиотеки \textit{React} и языка программирования \textit{TypeScript}, что обеспечило создание современного, высокодинамичного пользовательского интерфейса со строгой типизацией данных, минимизирующей количество ошибок времени выполнения.

Серверная часть системы (\textit{Backend}) была реализована на языке \textit{Kotlin} с использованием асинхронного фреймворка \textit{Ktor}. Это технологическое решение обеспечило надежность, масштабируемость и высокую скорость обработки \textit{REST API}-запросов. В качестве системы управления базами данных (СУБД) была использована \textit{PostgreSQL}, позволяющая эффективно хранить и управлять информацией о пользователях, сложной структуре заказов, парке принтеров и истории транзакций материалов.

В работе детально рассмотрена функциональная составляющая системы, включая реализацию всех \textit{CRUD}-операций и ключевых бизнес-процессов, таких как автоматический расчет и списание остатков филамента при выполнении заказа, управление очередью печати и формирование аналитической отчетности для менеджеров. Также описан принцип взаимодействия компонентов системы и механизм обеспечения безопасности данных посредством хеширования паролей и \textit{JWT}-авторизации.

Архитектура разработанного решения предусматривает возможность дальнейшего масштабирования, включая потенциальную интеграцию с мобильными платформами в рамках будущей дипломной работы.

Полученные в ходе выполнения работы практические навыки в проектировании архитектуры клиент-серверного приложения, работе с современным стеком технологий (\textit{Kotlin}, \textit{React}, \textit{TypeScript}) и управлении реляционными данными имеют высокую значимость и могут быть востребованы при решении аналогичных задач в профессиональной деятельности.

Пояснительная записка оформлена в соответствии с требованиями стандарта предприятия [3].