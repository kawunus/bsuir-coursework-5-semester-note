\section{Разработка АИС студии 3\textit{D}-печати}

\subsection{Выбор средств реализации}

Рассмотрим различные варианты, которые могут быть использованы для реализации данного проекта, с учетом специфики систем управления мелкосерийным производством, а также особенностей обработки и хранения данных в таком контексте. Важно учитывать, что в системах, связанных с аддитивным производством, часто требуется работать с большим объемом информации о заказах, номенклатуре материалов и состоянии оборудования, что накладывает определенные требования на выбор инструментов и технологий для эффективного управления данными.

Выбор подходящей системы управления базами данных (СУБД) играет решающую роль в создании эффективных и высокопроизводительных информационных систем. СУБД обеспечивают надежные инструменты для хранения, организации и управления данными, предоставляя быстрый доступ к информации о складских остатках и текущих процессах печати.

Существует несколько СУБД, которые могут быть использованы при разработке веб-приложений для автоматизации производства. Рассмотрим основные из них, которые часто применяются в таких проектах:

\textit{MySQL} – это одна из самых популярных СУБД с открытым исходным кодом. Она известна своей простотой в использовании, хорошей поддержкой и широким распространением. \textit{MySQL} отлично подходит для небольших веб-приложений, предлагая удобные возможности масштабирования [2].

\textit{PostgreSQL} выделяется мощными возможностями работы с более сложными запросами и транзакциями. Она имеет поддержку сложных структур данных, таких как \textit{JSON} для хранения гибкой информации о параметрах печати (настройках слайсера) или конфигурациях материалов, а также обладает высокой масштабируемостью и открытым исходным кодом. Это делает её отличным выбором для проектов со сложной бизнес-логикой.

\textit{SQLite} представляет собой встроенную СУБД, которая не требует отдельного сервера. Она лёгкая, простая в использовании и идеально подходит для прототипов или мобильных приложений, где нет необходимости в высокой конкурентной нагрузке [4].

\textit{MongoDB} – это гибридная СУБД \textit{NoSQL}, которая использует \textit{JSON}-подобные документы. Её основными преимуществами являются гибкость в схеме данных и высокая масштабируемость. \textit{MongoDB} подходит для хранения неструктурированных метаданных 3D-моделей, но может уступать реляционным СУБД в строгости связей [5].

\textit{Microsoft SQL Server} – это корпоративная СУБД с мощной поддержкой транзакций и хорошей интеграцией с другими продуктами \textit{Microsoft}. Она используется в больших корпоративных средах [6].

\textit{PostgreSQL} имеет несколько ключевых преимуществ, которые делают её особенно привлекательной для проекта АИС студии 3\textit{D}-печати. Она поддерживает сложные запросы, транзакции и обработку больших объёмов данных, что критично для проектов, требующих точного учета движения материалов (списания филамента), управления очередью печати и ведения истории статусов заказов. \textit{PostgreSQL} также предоставляет широкие возможности для работы с \textit{JSON}-данными, что полезно для хранения специфических атрибутов различных типов пластика без изменения схемы БД. Это делает её оптимальным выбором для проекта, который ожидает значительного масштабирования в будущем [8].

Выбор средств реализации клиентской части

Для разработки клиентской части системы было принято решение использовать современный стек технологий, базирующийся на библиотеке \textit{React} и языке программирования \textit{TypeScript}. Данный выбор обусловлен необходимостью реализации сложной бизнес-логики на стороне клиента и требований к надежности кода.

\textit{React} — это \textit{JavaScript}-библиотека для создания пользовательских интерфейсов, разработанная \textit{Facebook}. Её ключевой особенностью является компонентный подход, который позволяет разбивать сложные интерфейсы на изолированные, повторно используемые части (компоненты). Это критически важно для текущего проекта, так как система разделена на два независимых веб-приложения:
\begin{itemize}
    \item \textbf{Клиентский портал:} Публичный сайт для оформления заказов, ориентированный на максимальную простоту и \textit{UX}.
    \item \textbf{Панель администрирования:} Закрытое веб-приложение для сотрудников и аналитиков, насыщенное таблицами, графиками и элементами управления оборудованием.
\end{itemize}

Использование \textit{React} позволяет переиспользовать общие компоненты (например, кнопки, поля ввода, визуализацию статусов) в обоих приложениях, сокращая время разработки.

В качестве основного языка разработки выбран \textit{TypeScript}. Это надстройка над \textit{JavaScript}, добавляющая строгую статическую типизацию. В контексте АИС студии 3\textit{D}-печати использование \textit{TypeScript} обосновано следующими факторами:
\begin{itemize}
    \item \textbf{Безопасность данных:} Строгая типизация гарантирует, что структуры данных (например, объект \textit{Order} или \textit{Filament}), полученные с сервера, будут использоваться корректно. Это исключает ошибки, связанные с попыткой сложить вес филамента (число) с его названием (строка).
    \item \textbf{Поддержка рефакторинга:} При изменении структуры данных на бэкенде, \textit{TypeScript} автоматически подсветит все места в коде обоих клиентских приложений, требующие исправлений.
    \item \textbf{Читаемость кода:} Явное указание типов облегчает командную разработку и дальнейшую поддержку проекта.
\end{itemize}

Важно отметить, что в контексте веб-разработки часто возникает вопрос о соотношении \textit{React} и \textit{Node.js}. \textit{React} отвечает за визуализацию интерфейса (\textit{Frontend}), в то время как \textit{Node.js} используется как среда для сборки проекта и управления зависимостями через менеджеры пакетов (\textit{npm} или \textit{yarn}). В данном проекте среда \textit{Node.js} используется исключительно для инструментов разработки и сборки клиентских бандлов.

Традиционный стек \textit{HTML/CSS/JavaScript} без использования фреймворков был отвергнут, так как он не позволяет эффективно управлять состоянием приложения (state management) при сложной логике взаимодействия. Например, мгновенное обновление статуса принтера в панели администратора или динамический расчет стоимости заказа на клиентском сайте требовали бы написания большого объема императивного кода, сложного в поддержке.

Таким образом, выбор связки \textit{React} + \textit{TypeScript} является оптимальным для реализации архитектуры с двумя разделенными интерфейсами. Это обеспечивает высокую производительность, надежность и удобство сопровождения программного продукта, а также позволяет реализовать богатый функционал для разных ролей пользователей: от интуитивного мастера заказа для Клиента до нагруженного аналитического дашборда для Аналитика.

Выбор средств реализации серверной части

В процессе выполнения курсового проекта был проведен анализ инструментов для разработки серверной части (\textit{Backend}). В качестве основных альтернатив рассматривались классический подход на языке \textit{Java} с фреймворком \textit{Spring Boot} и современный подход на языке \textit{Kotlin} с использованием фреймворка \textit{Ktor} [9]. Оба варианта работают на виртуальной машине \textit{Java (JVM)}, что обеспечивает надежность и кроссплатформенность, однако они имеют существенные архитектурные различия.

\textit{Java} является отраслевым стандартом для создания крупных корпоративных систем. Фреймворк \textit{Spring Boot} предоставляет мощнейший функционал «из коробки», включая управление зависимостями и работу с БД. Однако для целей разрабатываемой АИС он обладает рядом недостатков: высокая ресурсоемкость, длительное время «холодного» старта и избыточная сложность конфигурации для проекта среднего масштаба.

\textit{Kotlin}, разработанный компанией \textit{JetBrains}, представляет собой современный статически типизированный язык, полностью совместимый с \textit{Java}, но лишенный её многословности. Ключевыми преимуществами \textit{Kotlin} для данного проекта являются:
\begin{itemize}
    \item \textbf{Null-safety:} Встроенная в систему типов защита от ошибок, связанных с пустыми ссылками (\textit{NullPointerException}), что критически важно для надежности расчетов стоимости и веса материалов.
    \item \textbf{Лаконичность:} Код на \textit{Kotlin} занимает значительно меньше места, оставаясь при этом читаемым и выразительным.
\end{itemize}

В качестве веб-фреймворка был выбран \textit{Ktor}. Это асинхронный фреймворк, созданный специально для \textit{Kotlin}. Его главное отличие от аналогов — использование \textbf{корутин} (\textit{Coroutines}). Это позволяет обрабатывать тысячи одновременных запросов (например, обновления статусов от множества клиентов и сотрудников) с минимальными затратами системных ресурсов, не блокируя потоки выполнения.

Для реализации архитектуры приложения также были выбраны следующие технологии, выявленные в ходе анализа кода:
\begin{itemize}
    \item \textbf{Koin:} Легковесный фреймворк для внедрения зависимостей (\textit{Dependency Injection}). В отличие от \textit{Spring}, он не использует рефлексию и прокси-объекты, что ускоряет запуск приложения и упрощает отладку [10].
    \item \textbf{JWT (JSON Web Token):} Стандарт для безопасной аутентификации пользователей без хранения сессий на сервере.
    \item \textbf{Exposed / Hibernate:} Для взаимодействия с базой данных \textit{PostgreSQL}.
\end{itemize}

Выбор связки \textit{Kotlin} + \textit{Ktor} обоснован необходимостью создания высокопроизводительной и масштабируемой системы. В отличие от \textit{Java/Spring}, \textit{Ktor} предоставляет разработчику полный контроль над конфигурацией и маршрутизацией (\textit{Routing}), а использование корутин обеспечивает высокую отзывчивость интерфейса. Кроме того, использование \textit{Kotlin} на бэкенде открывает перспективу использования технологии \textit{Kotlin Multiplatform} в будущем, что позволит переиспользовать бизнес-логику при разработке мобильного приложения в рамках дипломного проекта [8].