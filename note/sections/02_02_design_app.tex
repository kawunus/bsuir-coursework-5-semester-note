\subsection{Проектирование приложения}

Приложение построено по клиент-серверной архитектуре, где функциональные обязанности разделены между двумя основными компонентами: клиентской и серверной частями. Клиентская часть представляет собой пользовательский интерфейс, который обеспечивает взаимодействие пользователя с системой, в то время как серверная часть  отвечает за обработку запросов, выполнение бизнес-логики и управление данными приложения.

Клиентская часть собирает запросы пользователя (клиента, тренера или администратора) и передает их на сервер через сеть. Серверная часть, реализованная с использованием шаблона \textit{Entity}-\textit{Boundary}-\textit{Control}, обрабатывает эти запросы с помощью \textit{сontrol}-объектов, выполняет необходимые операции с данными в системе управления базами данных (СУБД) через \textit{entity}-объекты и отправляет результаты обратно на клиентскую сторону. Вся коммуникация между клиентом и сервером осуществляется с помощью протокола \textit{HTTP}, что обеспечивает стандартное и надежное взаимодействие.

Дополнительно, серверная часть, включающая \textit{control}-объекты для реализации бизнес-логики, может включать элементы кэширования для повышения производительности и быстродействия приложения. Также обязательна реализация механизмов аутентификации и авторизации по ролям для контроля доступа, что повышает безопасность системы. В процессе взаимодействия могут использоваться различные методы оптимизации, такие как сжатие данных или асинхронная обработка запросов, что способствует более эффективной работе с большим объемом информации, например, при загрузке полного расписания или аналитических отчетов.

\subsubsection{} Диаграмма вариантов использования

Диаграммы вариантов использования служат для отображения взаимосвязей между группами вариантов использования и участниками, которые принимают участие в процессе. Эти диаграммы помогают наглядно представить, как различные элементы системы взаимодействуют между собой, а также показывают, какие действия выполняются в рамках каждого варианта использования.

Основные элементы диаграммы включают участников и варианты использования. Участник – это роль или набор ролей, выполняемых субъектом при взаимодействии с системой или её компонентами (например, с подсистемами или классами). Участником может быть как человек, так и другая система, подсистема или класс, которые представляют внешние объекты или субъекты по отношению к сущности системы.

Вариант использования представляет собой описание группы действий, выполняемых в системе с точки зрения участника, и направленных на достижение определённого результата. Эти варианты описывают типичные взаимодействия между пользователем и системой, а также отображают внешний интерфейс системы. Они отвечают на вопрос, что система должна сделать, а не как это должно быть реализовано.

Диаграмма вариантов использования может включать дополнительные элементы, такие как системные границы, комментарии и атрибуты для вариантов использования. Также на диаграмме могут быть указаны различные типы связей, например, связь между участником и вариантом использования, которая демонстрирует, какие функции бизнес-системы могут быть использованы субъектом. 

Ассоциация – это связь между двумя элементами модели, например, между участником и вариантом использования. Она описывает причину связи и управляющие ею правила. Ассоциация указывает на то, что участник может использовать определённые функциональные возможности системы.

Отношение обобщения – это связь, в которой один элемент модели (дочерний) основан на другом (родительском). В контексте диаграмм вариантов использования отношение обобщения означает, что дочерний элемент наследует все атрибуты, операции и отношения родительского элемента. Это позволяет моделировать общие и частные варианты использования.

Отношение включения – это связь, в которой один вариант использования (базовый) включает в себя функциональные возможности другого варианта использования (вариант использования включения). Это отношение способствует повторному использованию функциональности, предотвращая дублирование описания схожих действий.

Отношение расширения – указывает на то, что один вариант использования (расширение) добавляет дополнительные действия или поведение к другому варианту использования (базовому). Это помогает раскрыть детали о системе, которые обычно скрыты в базовом варианте использования, расширяя функциональные возможности системы.

В АИС студии \textit{3D}-печати \textit{KawunPrint} можно выделить следующие действующие лица: неавторизованный пользователь, клиент, сотрудник, аналитик и администратор.

Исходя из потребностей действующих лиц, выделяются следующие варианты использования:
\begin{itemize}
    \item авторизоваться в системе;
    \item зарегистрироваться;
    \item ввести или обновить личные данные в профиле;
    \item подтвердить \textit{Email} с помощью проверочного кода;
    \item выйти из системы;
    \item создать заказ на \textit{3D}-печать;
    \item загрузить файлы \textit{3D}-моделей (интеграция с \textit{Firebase Storage});
    \item просмотреть список и историю своих заказов;
    \item просмотреть детализацию изменений статусов заказа;
    \item просмотреть список всех заказов (для сотрудника и аналитика);
    \item изменить статус заказа (например, перевести в «Печать» или «Готов»);
    \item списать расходные материалы (\textit{Filament}) под конкретный заказ;
    \item добавить или удалить тип пластика;
    \item управление запасами филамента (добавление катушек, учет остатков);
    \item управление парком принтеров (добавление, редактирование, списание);
    \item изменить статус активности принтера (включен/выключен);
    \item вести журнал истории обслуживания принтеров;
    \item просмотреть список всех зарегистрированных пользователей;
    \item назначить роль пользователю (\textit{Admin}, \textit{Employee}, \textit{Analyst});
    \item просмотреть финансовую статистику и отчеты;
    \item просмотреть статистику загрузки файлов и использования дискового пространства;
    \item прямое администрирование файлов в облачном хранилище;
    \item удалить пользователя или ошибочные данные из БД.
\end{itemize}

Диаграмма вариантов использования АИС студии \textit{3D}-печати представлена на рисунке \ref{fig:usecase}.

\begin{figure}[H]
    \centering
    \includegraphics[width=0.62\textwidth]{attachments/usecade_diagram.png}
    \caption{Диаграмма вариантов использования АИС студии \textit{3D}-печати}
    \label{fig:usecase}
\end{figure}

На диаграмме вариантов использования для базы студии \textit{3D}-печати показано взаимодействие между вариантами использования и действующими лицами. Диаграмма отражает требования к системе, демонстрируя, какие действия должны выполняться различными пользователями для эффективного взаимодействия с базой данных студии.

Для каждого действующего лица выделяются определённые варианты использования, в зависимости от их потребностей и ролей в системе. Например, неавторизированный пользователь может зарегистрироваться или авторизоваться, в то время как авторизированный пользователь имеет доступ к просмотру и изменению  личных данных, созданию заказов и их просмотру. Сотрудники и аналитики имеют расширенные возможности, такие как управление заказами, списание материалов и просмотр статистики. Администраторы обладают полным доступом к управлению пользователями и данными системы.

Эта диаграмма служит ключевым инструментом для проектирования базы данных, чётко структурируя все требования к системе и показывая, какие функции должны быть доступны различным ролям в системе.

\subsubsection{} Диаграмма классов

Диаграмма классов для проектируемой автоматизированной
информационной системы управления абонентами изображена на
рисунке \ref{fig:class_diagram}.

\begin{figure}[H]
    \centering
    \includegraphics[width=1\textwidth]{attachments/class_diagram.png}
    \caption{Диаграмма классов}
    \label{fig:class_diagram}
\end{figure}

Атрибуты характеризуют свойства объектов класса. Большинство
объектов становятся уникальными благодаря различиям в их атрибутах и
взаимосвязям с другими объектами.

В диаграмме классов также отображаются различные виды связей
между классами.

Ассоциация – представляет собой отношение между экземплярами
классов. Каждый конец ассоциации указывает кратность, которая показывает,
сколько объектов с соответствующего конца могут быть вовлечены в
данное отношение. Ассоциации могут быть названы, и обычно для этого выбирается глагол или
глагольная фраза, которая объясняет смысл и цель связи.

Агрегация – это тип ассоциации "целое-часть". В \textit{UML} агрегация
изображается в виде прямой линии с ромбом на одном из концов. Композиция
является разновидностью агрегации, при которой объекты-части не
могут существовать отдельно от объекта, который их агрегирует. Если
агрегирующий объект уничтожается, то уничтожаются и все части,
связанные с ним. Композиция отображается
так же, как агрегация, но ромб при этом закрашен.

Наследование – это отношение "общее-частное". Оно позволяет
создавать иерархию классов, где один класс наследует поведение и
структуру других классов. Наследование является основой для повторного
использования кода и достижения полиморфизма, что является важным
инструментом в объектно-ориентированном программировании. Когда
создается производный класс на основе базового, образуется иерархия,
которая позволяет эффективно использовать общие компоненты кода
в разных контекстах.

Эти элементы и связи на диаграмме классов помогают чётко
структурировать систему и моделировать её поведение и отношения между
объектами.



\subsubsection{} Схемы алгоритмов

Схемы алгоритмов, или блок-схемы (\textit{Flowcharts}), представляют собой фундаментальный графический инструмент в области информатики, инженерии и управления процессами. Они служат для наглядного и недвусмысленного отображения логической структуры алгоритма или последовательности действий, которые необходимо выполнить для решения конкретной задачи или достижения поставленной цели.

Алгоритм авторизации (\textit{/api/v1/login}) начинается с получения данных, содержащих \textit{email} и пароль, в виде объекта \textit{LoginRequest}.

Если данные (\textit{LoginRequest}) не получены или не распознаны корректно, алгоритм выводит ошибку (400 \textit{Bad Request}) и завершается.

Если данные получены верно, система формирует запрос на проверку существования пользователя по его \textit{email} в таблице \textit{Users}.

Если результат запроса отрицательный (пользователь не найден), то система выводит соответствующую ошибку (400: Неверный \textit{Email}), и алгоритм заканчивается.

Если же результат положительный (пользователь найден), то система инициирует проверку совпадения хешей паролей.

После этого проверяется, совпали ли хеши.

Если нет, то система выводит ошибку (400: Неверный Пароль), и процесс завершается.

Если да, то система генерирует авторизационный \textit{JWT} токен.

После этого проверяется, завершены ли все операции успешно (в рамках \textit{try}-\textit{catch} блока).

Если да, то выводится сообщение от системы об успешной авторизации (\textit{200 OK}) с возвратом токена.

После этого процесс завершается.

Алгоритм авторизации представлен на рисунке \ref{fig:auth_algorithm}.

\begin{figure}[H]
    \centering
    \includegraphics[width=0.37\textwidth]{attachments/auth_algorithm.png}
    \caption{Схема алгоритма авторизации в системе}
    \label{fig:auth_algorithm}
\end{figure}

Алгоритм был сделан таким образом, чтобы подходить сразу и для администраторской части веб-преложения, и для клиентской части веб-приложения.

Алгоритм подтверждения \textit{email} начинается с запроса на отправку кода верификации (\textit{/send}-\textit{verification}-\textit{code}).

\begin{itemize}
    \item Система получает \textit{email} и проверяет существование пользователя в \textit{Users}.
    \item Если \textit{email} не найден, то система выводит ошибку (400 \textit{Bad Request}) и процесс завершается.
    \item Если \textit{email} найден, система генерирует уникальный код, удаляет все старые коды для этого пользователя из \textit{EmailVerificationCodeTable} и записывает новый код с меткой времени истечения (15 минут).
    \item Система пытается отправить код на \textit{email} пользователя с помощью \textit{EmailService}.
    \item После этого проверяется, успешно ли отправлено сообщение.
    \item Если да, то выводится сообщение об успешной отправке (200 \textit{OK}). Если нет, выводится ошибка (500 \textit{Internal Server Error}), и процесс завершается.
\end{itemize}

После этого начинается второй этап алгоритма – подтверждение кода верификации (\textit{/verify}-\textit{email}).

\begin{itemize}
    \setcounter{enumi}{6}
    \item Система получает \textit{email} и введенный код.
    \item Система проверяет существование пользователя и запрашивает из \textit{EmailVerificationCodeTable} код, который совпадает, принадлежит пользователю и не просрочен.
    \item Если код недействителен или просрочен, то система выводит соответствующую ошибку (400 \textit{Bad Request}), и алгоритм заканчивается.
    \item Если код действителен, система инициирует транзакцию и устанавливает флаг \textit{isActive} = \textit{true} для пользователя в таблице \textit{Users}, а затем удаляет все коды верификации для этого пользователя.
    \item После активации пользователя система получает его обновленные данные.
    \item Если пользователь успешно получен, генерируется новый \textit{JWT} токен, содержащий \textit{isActive} = \textit{true}.
    \item Выводится сообщение об успешной верификации (200 \textit{OK}) с возвратом нового \textit{JWT} токена.
    \item После этого процесс завершается.
\end{itemize}

Схема алгоритма подтверждения \textit{email} представлен на рисунке \ref{fig:email_verification_algorithm}.

\begin{figure}[H]
    \centering
    \includegraphics[width=0.8\textwidth]{attachments/verifier_algorithm.png}
    \caption{Схема алгоритма подтверждения \textit{email} в системе}
    \label{fig:email_verification_algorithm}
\end{figure}

Эти схемы алгоритмов способствуют упрощению процесса разработки и обеспечивают эффективную реализацию функциональных требований.