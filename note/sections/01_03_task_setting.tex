\subsection{Постановка задачи}
\label{sec:analysis:specification}

Исходя из анализа веб-приложений и управляющих систем в сфере аддитивных технологий, сделаны следующие выводы. Необходимо провести проектирование элементов системы таким образом, чтобы функционал равномерно удовлетворял требованиям всех пользователей: клиентов, сотрудников, аналитиков и администраторов.

Основной задачей, поставленной в ходе курсового проектирования, является проектирование и разработка базы данных и серверного приложения (\textit{backend}), а также клиентского интерфейса. Решение перечисленных задач позволит достигнуть цели, поставленной в курсовом проекте, а именно — создать АИС студии \textit{3D}-печати.

\subsubsection{} Требования к безопасности. 

Необходимо учитывать требования к безопасности данных, гарантируя защиту конфиденциальности и целостности информации пользователей. Для обеспечения безопасности пароли всех пользователей должны проходить процесс хеширования перед сохранением в базе данных. Этот шаг направлен на защиту от компрометации учетных записей.

Важным аспектом является обеспечение доступа только авторизованным пользователям, что реализовано с помощью системы ролей (\textit{Role-Based Access Control}). Это гарантирует, что только пользователи, имеющие соответствующие полномочия (\textit{Client, Employee, Analyst, Admin}), смогут получить доступ к критическим данным и функциям, таким как списание материалов или изменение конфигурации принтеров.

\subsubsection{} {Функциональные модули.} 

В качестве реализации функционала системы должны быть предусмотрены модули для каждой роли:

\begin{itemize}
    \item Для {Администратора} и {Сотрудника} должна отображаться информация об очереди заказов, клиентах, запасах филамента и состоянии принтеров.
    \item Для {Аналитика} должен быть доступен модуль визуализации данных, реализующий отображение графиков продаж, расхода материалов и эффективности сотрудников.
    \item Для {Клиента} — интерфейс оформления заказа с загрузкой параметров, а также раздел с историей и статусами выполнения.
\end{itemize}

\subsubsection{} Архитектура проектируемой системы закладывается с учетом возможности дальнейшего расширения. В частности, использование \textit{REST API} на базе фреймворка \textit{Ktor} позволяет в будущем, в рамках дипломного проектирования, разработать кроссплатформенное мобильное приложение (для \textit{iOS} и \textit{Android}), использующее уже существующий бэкенд и модели. Это обеспечит омниканальный доступ к услугам студии без необходимости переработки серверной части.

Аппаратная платформа относится к физическим компонентам системы, которые необходимы для запуска программного обеспечения. Требования к серверной части включают наличие достаточных вычислительных мощностей для обработки запросов к базе данных и выполнения бизнес-логики.

Для устройств (клиентских и рабочих станций сотрудников), на которых будет работать веб-приложение, рекомендуется использовать процессор с частотой не менее 1.6 ГГц, ОЗУ 4 ГБ и более, место на жестком диске (или SSD) не менее 20 ГБ. Видеоадаптер должен поддерживать современные стандарты вывода графики для корректного отображения интерфейса и (в перспективе) предпросмотра \textit{3D}-моделей. Эти требования направлены на обеспечение оптимальной производительности и отзывчивости интерфейса.

\subsubsection{} Программная платформа относится к совокупности программного обеспечения, обеспечивающего среду запуска.

Серверная часть приложения разрабатывается для кроссплатформенной среды \textit{JVM}, что позволяет развертывать её на операционных системах \textit{Linux} (ядро 4.15+), \textit{Windows Server} или \textit{macOS}.

Клиентская часть (веб-интерфейс) поддерживает работу в современных браузерах: \textit{Google Chrome}, \textit{Mozilla Firefox}, \textit{Safari} или \textit{Microsoft Edge}. Требуется поддержка стандартов \textit{HTML5}, \textit{CSS3} и \textit{JavaScript (ES6+)} для обеспечения корректной работы динамических элементов управления заказами и графиков аналитики.

