\section{Анализ предметной области}

\subsection{Описание предметной области}

Процесс эффективного управления деятельностью современной студии \textit{3D}-печати принадлежит к ряду нетривиальных инженерно-экономических задач, особенно без наличия структурированной информационной системы для контроля за производственными и логистическими цепочками. С технической точки зрения сложность задачи управления студией аддитивных технологий сводится к синхронизации потоков цифровой информации (файлы моделей, параметры слайсинга) и физических ресурсов (парк принтеров, расходные материалы, постобработка).

Учет реальных остатков филамента с точностью до грамма, грамотный менеджмент очереди печати и мониторинг технического состояния оборудования (\textit{printers}) с целью минимизации простоев является живой, востребованной к решению в индустрии \textit{additive manufacturing} задачей. Хаотичность поступления заказов, разнородность форматов файлов (\textit{STL, OBJ, 3MF}), несинхронизированная информация о наличии конкретных цветов пластика и отсутствие контроля за сроками выполнения приводит к потере репутации и финансовым убыткам. Оптимизация внутренних процессов обработки клиентских запросов и учета ресурсов позволяет увеличивать доход системы мелкосерийного производства.

Рассматривая предметную область, необходимо затронуть исторический аспект формирования рынка услуг \textit{3D}-печати. Изначально, с появлением доступных \textit{FDM}-принтеров (проект \textit{RepRap}), услуги печати носили кустарный характер. Взаимодействие строилось на личных договоренностях на форумах энтузиастов.

С развитием технологий и появлением профессиональных настольных решений (таких как \textit{Bambu Lab, Prusa, Ultimaker}), студии начали масштабироваться в так называемые «печатные фермы» (\textit{print farms}). Однако методы управления долгое время оставались архаичными. Эволюция процессов прошла путь от записи заказов в бумажный блокнот до использования разрозненных таблиц в \textit{Excel} или \textit{Google Sheets}.

На ранних этапах отсутствовало понятие унифицированного «типа филамента» (\textit{Filament Type}). Оператор мог использовать любой пластик, визуально похожий на требуемый, что приводило к браку из-за неверных температурных режимов. История обслуживания принтеров не велась вовсе: ремонт осуществлялся только по факту поломки, что парализовывало выполнение заказов в пиковые периоды нагрузки. Отсутствие системного подхода к ценообразованию (расчет стоимости на основе времени печати или веса) часто приводило к работе в убыток, так как не учитывались амортизация и стоимость редких композитных материалов.

До внедрения автоматизированных информационных систем (АИС), управление студией \textit{3D}-печати базируется на ручном труде и разрозненных программных инструментах. Рассмотрим типичный сценарий работы «неавтоматизированной» студии, чтобы выделить проблематику.

\begin{itemize}
    \item \textbf{Прием заказа:} Клиент отправляет файл модели через сайт (\textit{Telegram, WhatsApp}) или электронную почту. Администратор должен вручную скачать файл, открыть его в программе-слайсере (\textit{Slicer}), выставить настройки, рассчитать вес и время, а затем написать клиенту цену. Это занимает значительное время.
    \item \textbf{Учет материалов:} Информация о катушках с пластиком хранится либо в голове сотрудника, либо на стикерах, наклеенных на сами катушки. Часто возникает ситуация, когда в таблице указано наличие «Синего \textit{PLA}», но по факту на катушке осталось 50 грамм, чего недостаточно для заказа. Оператор узнает об этом только в момент заправки принтера.
    \item \textbf{Управление очередью:} Очередь печати формируется хаотично. Сотрудник выбирает, что печатать, исходя из субъективных предпочтений, а не срочности. Нет четкой привязки заказа к конкретному принтеру до момента запуска.
    \item \textbf{История и статистика:} Без АИС невозможно получить аналитику: какой тип пластика самый популярный, какой принтер ломается чаще всего, какова эффективность конкретного сотрудника (\textit{Employee}). История заказов (\textit{Order History}) представляет собой архив переписок в почте, в котором крайне сложно ориентироваться.
\end{itemize}

В таких условиях риск человеческой ошибки (запуск печати не тем цветом, потеря заказа, пропуск дедлайна) становится критическим фактором, ограничивающим рост бизнеса.

В контексте реализации системы базы данных АИС студии \textit{3D}-печати выделяется ряд ключевых бизнес-процессов, подлежащих автоматизации:

\begin{enumerate}[label=\arabic*]
    \item \textbf{Управление номенклатурой материалов (Filaments):}
    Организация строгого учета расходных материалов. Система должна хранить данные о каждой конкретной катушке пластика (\textit{FilamentModel}): его цвет (\textit{hexColor}), тип (\textit{FilamentType}), стоимость за грамм (\textit{pricePerGram}) и, что наиболее важно, текущий остаток (\textit{residue}). Процесс списания материала должен быть интегрирован в процесс выполнения заказа.

    \item \textbf{Управление парком оборудования (Printers):}
    Учет физических единиц оборудования. Каждый принтер имеет статус активности (\textit{active state}) и историю обслуживания (\textit{PrinterHistory}). Это позволяет распределять нагрузку: система не должна позволять назначать заказ на неисправный или занятый принтер.

    \item \textbf{Жизненный цикл заказа (Order Lifecycle):}
    Оценка и организация процесса прохождения заказа от заявки до выдачи. Заказ проходит через ряд статусов (\textit{OrderStatus}), которые изменяются сотрудниками или администраторами. Важнейшим аспектом является сохранение истории изменений заказа для разрешения спорных ситуаций.
    
    \item \textbf{Аналитика и отчетность (Analytics \& Reporting):}
    Сбор и визуализация данных для Аналитика (\textit{Analyst}). Это включает построение графиков продаж, анализ популярности типов пластика и оценку загруженности сотрудников. Данный процесс необходим для корректировки стратегий развития студии.

    \item \textbf{Взаимодействие с пользователями (Users \& Roles):}
    Организация базы данных пользователей с разделением прав доступа (\textit{RoleModel}). Клиенты должны иметь возможность самостоятельно отслеживать статус своих заказов, сотрудники (\textit{Employee}) — управлять производством, а аналитики (\textit{Analyst}) — просматривать отчеты без права вмешательства в технологический процесс.
\end{enumerate}

Поставка услуг в виде готовых изделий осуществляется за счет внутреннего производственного менеджмента. Управление производством формируется путем сопоставления параметров заказа (материал, цвет) с имеющимися ресурсами. Осуществление предоставления услуг реализуется путем своевременного назначения задач на свободные принтеры и контроля расхода материала.

Среди способов взаимодействия клиента с услугами студии можно выделить ряд пунктов:
\begin{itemize}
    \item \textbf{Онлайн-заказ:} Оформление заявки через веб-приложение с прикреплением параметров заказа.
    \item \textbf{Прямое взаимодействие:} Оформление заказа через администратора (по телефону или лично), который создает запись в системе от имени клиента, используя свою роль сотрудника.
    \item \textbf{Стратегическое управление:} Доступ аналитика к «Дашборду» (панели управления) с графиками выручки и расхода материалов для принятия управленческих решений.
    \item \textbf{Мониторинг:} Просмотр истории заказов и статусов в личном кабинете (\textit{User Profile}).
\end{itemize}

Следующим немаловажным бизнес-процессом является организация рабочей зоны. Необходимо осуществить цифровую маркировку принтеров. В базе данных каждый принтер имеет уникальный \textit{ID}, который физически нанесен на устройство. Это позволяет сотруднику при возникновении поломки или завершении печати точно идентифицировать устройство в системе и внести запись в журнал (\textit{Printer History}).

Также критически важна организация хранения филамента. Каждая катушка в системе уникальна. При получении заказа на печать «Красным \textit{PETG}», сотрудник должен выбрать в системе конкретную катушку с достаточным остатком (\textit{residue}) и зарезервировать необходимое количество грамм.

Оценка и организация процесса выполнения заказа является весьма нелинейной задачей, требующей к реализации приложения профессионального понимания технологии печати.

Стоит привести обобщенную процедуру обработки заказа в рамках разрабатываемой АИС.

Прежде чем оформить заказ, клиенту (или администратору от его имени) необходимо пройти авторизацию. Система проверяет роль пользователя (\textit{Auth with Role}). После успешного входа пользователь формирует заявку (\textit{CreateOrderRequest}).

Когда заказ поступает в систему, начинается процесс его обработки сотрудником:
\begin{enumerate}[label=\arabic*]
    \item \textbf{Валидация и оценка:} Сотрудник проверяет возможность печати модели и выбирает подходящий материал из доступных в базе (\textit{FilamentUseCase}).
    \item \textbf{Производство:} Заказ назначается на принтер. В этот момент сотрудник может изменить статус заказа на «В работе».
    \item \textbf{Списание материала (Consume Filament):} Это критический этап автоматизации. По завершении печати (или перед ней) сотрудник вносит данные о фактически затраченном весе пластика. Система автоматически вычитает этот вес из остатка конкретной катушки (\textit{Update Filament Residue}). Если остаток становится меньше необходимого, система должна сигнализировать о невозможности выполнения операции или необходимости замены катушки.
    \item \textbf{Завершение:} После успешной печати и постобработки статус заказа меняется на «Готов», а дата завершения (\textit{completedAt}) фиксируется автоматически.
\end{enumerate}

Параллельно с этим процессом ведется журнал истории заказа (\textit{OrderHistory}), куда автоматически записывается: кто изменил статус, когда это произошло и какие комментарии были оставлены.

Важным аспектом функционирования АИС является обеспечение целостности данных и разграничение ответственности. В системе, где наряду с производственным персоналом (\textit{Employee}) работает аналитический персонал (\textit{Analyst}), критически важно разделить потоки данных. Аналитик должен иметь доступ к агрегированной информации (например, «Общая выручка за месяц»), но не должен иметь возможности случайно списать материал или удалить принтер. Реализованная ролевая модель (\textit{RoleModel}) и логика записи истории событий гарантируют, что каждый пользователь действует строго в рамках своих компетенций.

Таким образом, предоставление пользователям и сотрудникам актуальной справочно-аналитической информации по материалам, статусам принтеров и истории заказов является гарантом успешного функционирования студии. Единожды инвестированные средства на разработку качественного \textit{backend}-решения на \textit{Ktor} поспособствуют оптимизации бизнес-процесса ведения учета, позволят избежать коллизий с нехваткой материалов и повысят лояльность клиентов за счет прозрачности и скорости обслуживания.