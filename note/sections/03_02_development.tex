\subsection{Описание реализации базы данных}

Физическая модель базы данных представляет собой логическую модель, выраженную с использованием языка описания данных (\textit{DDL}) конкретной системы управления базами данных (СУБД). Она включает в себя все элементы, необходимые для создания базы данных в рамках выбранной СУБД, такие как наименования таблиц, столбцов, типы данных, а также определения первичных и внешних ключей.

В физической модели базы данных задаются ограничения для столбцов. К таким ограничениям относятся первичные и внешние ключи, уникальные ограничения, а также ограничения диапазонов значений столбцов. Эти механизмы направлены на обеспечение целостности данных и поддержание связей между таблицами, что важно для корректного функционирования базы данных АИС студии \textit{3D}-печати.

Существенной частью физической модели является размещение данных на физическом носителе, например, на жестком диске. Данные организуются в файлы, разделы или пространства таблиц, что позволяет оптимизировать их физическое распределение. Это включает размещение таблиц и индексов, выбор методов хранения данных и их структуры с учетом предполагаемой нагрузки и типов запросов. 

Разработка физической схемы базируется на логической схеме и осуществляется с учётом требований к производительности, доступности и эффективности обработки данных. Оптимизация физической модели проводится с учётом возможностей конкретной СУБД и характеристик используемого аппаратного обеспечения.

В рамках данного проекта для реализации базы данных использовалась СУБД \textit{PostgreSQL}.

Физическая схема базы данных представлена на рисунке \ref{fig:phisical_scheme}

\begin{figure}[H]
    \centering
    \includegraphics[width=0.65\textwidth]{attachments/physical_scheme.png}
    \caption{Физическая схема базы данных}
    \label{fig:phisical_scheme}
\end{figure}

Разработанная схема включает таблицы с атрибутами, которые отражают сущности, связанные с процессом управления студией \textit{3D}-печати (пользователи, заказы, принтеры, материалы). В каждой таблице выделены атрибуты, которые служат уникальными идентификаторами конкретных записей и выполняют роль первичных ключей (\textit{PK — Primary Key}). Для связывания таблиц между собой используются внешние ключи (\textit{FK — Foreign Key}).

Кроме того, при проектировании была применена нормализация данных для устранения избыточности и обеспечения целостности. Нормализация базы данных — это систематический процесс структурирования таблиц для минимизации дублирования информации, устранения аномалий (при операциях вставки, обновления и удаления) и повышения надежности данных. Этот процесс основан на соблюдении ряда правил, называемых нормальными формами:

\begin{itemize}
    \item Первая нормальная форма (1НФ): Требует, чтобы каждый столбец содержал только атомарные (неделимые) значения, и в таблице отсутствовали повторяющиеся группы атрибутов. Например, список заказанных услуг не хранится в одной ячейке, а разнесен по структуре.
    \item Вторая нормальная форма (2НФ): Помимо требований 1НФ, требует устранения частичных функциональных зависимостей: каждый неключевой атрибут должен полностью зависеть от всего первичного ключа таблицы.
    \item Третья нормальная форма (3НФ): Требует, чтобы таблица находилась во 2НФ и не содержала транзитивных зависимостей. Это означает, что неключевые атрибуты (например, описание типа пластика) не должны зависеть от других неключевых атрибутов внутри таблицы конкретного материала, а должны быть вынесены в справочник.
\end{itemize}

Достижение уровня 3НФ считается оптимальным для задач автоматизации производства, так как обеспечивает баланс между скоростью выборки и отсутствием дублирования.

Вся схема построена на неидентифицирующих связях, где внешний ключ в дочерней таблице не является частью ее первичного ключа. Это обеспечивает архитектурную гибкость и независимость каждой сущности, позволяя, например, независимо управлять жизненным циклом заказов (\textit{Orders}) и данными о пластике (\textit{Filament Types}).